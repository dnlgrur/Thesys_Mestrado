%%%%%%%%%%%%%%%%%%%%%%%%%% Configuração: frontmatter %%%%%%%%%%%%%%%%%%%%%%%%%%
\appto\frontmatter{\pagestyle{plain}}  % Adiciona o estilo plano de página.


%%%%%%%%%%%%%%%%%%%%%%%%%% Configurações: referências %%%%%%%%%%%%%%%%%%%%%%%%%%
%\addbibresource{tese.bib, Mendeley.bib}


%%%%%%%%%%%%%%%%%%%%%%%%%%%%% Configurações: links %%%%%%%%%%%%%%%%%%%%%%%%%%%%%
\hypersetup{
% TODO Por padrão os links, no pdf, para equações, figuras, referencias,
% tabelas, urls são identificados por uma caixa colorida em volta do link. Essa
% caixa colorida não eh impressa mas pode atrapalhar a leitura para alguns. Se
% desejar removê-las descomente a linha abaixo.
% hidelinks,
hypertexnames=false,
pdftitle={\titulo},  % Não modifique esta linha.
pdfauthor={\autor}  % Não modifique esta linha.
}


%%%%%%%%%%%%%%%%%%%%%%%%%% Configurações: numeração %%%%%%%%%%%%%%%%%%%%%%%%%%
\numberwithin{equation}{section}
\numberwithin{section}{chapter}


%%%%%%%%%%%%%%%%%%%%%%%%%%%% Configurações: amsthm %%%%%%%%%%%%%%%%%%%%%%%%%%%%
% Altera o estilo dos Teoremas, Conjecturas, ... e os possíveis valores
% são: plain, definition e remark.
% \theoremstyle{definition}
% Definição dos Teoremas, Conjecturas, ... e numeração dos mesmos
\newtheorem{thm}{Teorema}[section]
\newtheorem{con}[thm]{Conjectura}
\newtheorem{cor}[thm]{Corolário}
\newtheorem{dfn}[thm]{Definição}
\newtheorem{exm}[thm]{Exemplo}
\newtheorem{lem}[thm]{Lema}
\newtheorem{obs}[thm]{Observação}
\newtheorem{pps}[thm]{Proposição}


%%%%%%%%%%%%%%%%%%%%%%%%%%%%%% Configurações: códigos %%%%%%%%%%%%%%%%%%%%%%%% 
\lstset{
basicstyle=\ttfamily,
keywordstyle=\bfseries\color{green!40!black},
commentstyle=\color{gray},
stringstyle=\color{Maroon},
identifierstyle=\color{Blue},
numbers=left,
numberstyle=\tiny,
breaklines=false
}


%%%%%%%%%%%%%%%%%%%%%%%%%%%%% Configurações: anexo %%%%%%%%%%%%%%%%%%%%%%%%%%%
\newcommand{\annexname}{Anexo}
\makeatletter
\newcommand\annex{\par
\setcounter{chapter}{0}%
\setcounter{section}{0}%
\gdef\@chapapp{\annexname}%
\gdef\thechapter{\@Roman\c@chapter}}
\makeatother


\usepackage{color, colortbl}
\usepackage{xcolor}
\usepackage{tikz}
\usetikzlibrary{tikzmark}
\definecolor{LRed}{rgb}{1,.8,.8}
\definecolor{MRed}{rgb}{1,.6,.6}
\definecolor{HRed}{rgb}{1,.2,.2}

\usepackage{acro} 
% TODO Inserir configurações adicionais aqui.